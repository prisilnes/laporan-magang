\begin{center}
    \textbf{IMPLEMENTASI FRAMEWORK HUGO \\
    PADA WEBSITE PROFIL PERUSAHAAN \\
     (STUDI KASUS: PT. BOGA PANGAN SENTOSA)}
    \textbf{ABSTRAK}
    \addcontentsline{toc}{section}{ABSTRAK}
\end{center}

Perkembangan teknologi terutama \emph{internet} dan sistem online akhir-akhir ini sangat pesat 
yang menyebabkan persaingan usaha semakin ketat. Hampir semua orang yang mempunyai komputer 
dan \emph{smart-phone} yang terkoneksi dengan \emph{internet}, oleh karena itu \emph{website} perusahaan harus berisi 
profil perusahaan dan jasa atau produk yang disediakan menjadi langkah awal agar memenangkan persaingan. 
\emph{website} memiliki peran penting dalam suatu bisnis seperti meningkatkan pengetahuan dan kepercayaan pembeli 
terhadap perusahaan. PT. Boga Pangan Sentosa (PT. BPS) merupakan perusahaan yang bergerak di bidang katering 
makanan untuk perusahaan industri. Namun, PT. BPS masih belum menerapkan strategi \emph{branding} yang efektif. 
Dengan adanya \emph{website} profil perusahaan, dapat meningkatkan reputasi dari perusahaan agar menjadi daya tarik 
pembeli dan membangun jaminan kualitas. \emph{website} profil perusahaan akan dikembangan dengan \emph{framework} Hugo. 
Hugo adalah situs \emph{statis generator} yang ditulis dengan bahasa Go. Hugo memiliki performa kecepatan \emph{website} 
dalam menyajikan konten, kemudahan dalam memperbaiki dan memperbaharui \emph{website} dikemudian hari, tampilan yang 
minimalis, serta \emph{website} dapat ditemukan dengan mudah pada \emph{search engine}.  Situs \emph{statis generator} 
fokus pada 
pekerjaan utama yaitu menghasilkan situs berbasis HTML yang tidak bergantung pada \emph{database} atau sumber data 
eksternal lainnya karena menghindari pemrosesan sisi server saat mengakses \emph{website}. 
Diharapkan dengan tersedianya \emph{website} PT. BPS ini diharapkan dapat meningkatkan kehadirannya, 
karena calon pelanggan dapat melihat informasi perusahaan secara detail kapan saja (365d x 24h). 
Oleh karena itu, \emph{website} yang dibuat harus berisi informasi lengkap dan menarik pelanggan serta dapat 
dengan kecepatan yang tinggi. 

\noindent Kata Kunci: \emph{Framework} Hugo, Go Lang, Situs \emph{Statis Generator}, \emph{website} Profil Perusahaan.
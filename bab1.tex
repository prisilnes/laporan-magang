\mychapter{1}{BAB I \\ PENDAHULUAN}


\section{Latar Belakang}

Perkembangan teknologi terutama \emph{internet} dan sistem \emph{online} akhir-akhir ini sangat pesat 
yang menyebabkan persaingan usaha semakin ketat. Pengusaha harus mempunyai suatu strategi yang inovatif, 
yaitu menerapkan teknologi informasi untuk memenangkan persaingan agar menjadi pemimpin pasar \emph{(market leader)}.

Langkah awal dalam penerapan teknologi ini adalah dengan membuat sebuah \emph{website} perusahaan, 
karena masyarakat sudah terbiasa untuk mencari informasi seperti produk dan jasa melalui \emph{internet}. 
Hampir semua orang yang mempunyai komputer dan \emph{smart-phone} yang terkoneksi dengan \emph{internet}, oleh karena itu 
\emph{website} perusahaan berisi profil perusahaan dan jasa / produk yang disediakan 
menjadi langkah awal agar memenangkan persaingan.

Salah satu langkah yang dapat dilakukan untuk dapat bersaing dalam dunia bisnis yaitu dengan membuat suatu \emph{website} perusahaan. 
\emph{website} memiliki peran penting dalam suatu bisnis 
seperti meningkatkan pengetahuan dan kepercayaan pembeli terhadap perusahaan. 
Dengan kata lain, \emph{website} banyak dilakukan untuk meningkatkan \emph{brand} image perusahaan. 
Informasi dapat diakses dimana saja dan kapan saja  selama ada koneksi \emph{internet}. 
Biaya pemasaran lebih murah dibandingkan dengan cara tradisional yang dahulu mencetak brosur dan sebagainya.

PT. Boga Pangan Sentosa (PT. BPS) merupakan perusahaan yang bergerak di bidang katering makanan untuk perusahaan industri. 
Namun, PT. BPS masih belum menerapkan strategi branding yang efektif karena sampai saat ini dilakukan dengan cara kuno. 
Branding bertujuan untuk membangun citra dari perusahaan. Dengan adanya \emph{website} profil perusahaan, 
dapat meningkatkan reputasi dari perusahaan agar menjadi daya tarik pembeli dan membangun jaminan kualitas.

Dalam pembuatan \emph{website}, diperlukan penelitian untuk menentukan \emph{Content Management System} (CMS) 
yang tepat dan sesuai dengan kriteria. Hal yang menjadi perbandingan seperti performa kecepatan \emph{website} 
dalam menyajikan konten, kemudahan dalam memperbaiki dan memperbaharui \emph{website} dikemudian hari, 
tampilan yang minimalis, dan \emph{website} dapat ditemukan dengan mudah pada \emph{search engine}. 
Dalam hal ini, yang memenuhi semua kriteria tersebut dengan menggunakan \emph{framework} Hugo. 
Hugo salah satu generator situs statis \emph{open-source} yang cukup populer.  
Generator situs statis fokus pada pekerjaan utama yaitu menghasilkan situs berbasis HTML
yang tidak bergantung pada \emph{database} atau sumber data eksternal lainnya karena menghindari 
pemrosesan sisi server saat mengakses \emph{website}.


Berdasarkan hal tersebut, BPS membutuhkan /emph{programmer} 
untuk membuat \emph{website} profil perusahaan 
untuk meningkatkan kepercayaan pelanggan dan /emph{brand} dari perusahaan. 
Dengan kesempatan yang ditawarkan, penulis memilih PT. BPS menjadi tempat praktek kerja magang.

Diharapkan dengan tersedianya \emph{website} PT. BPS ini 
diharapkan dapat meningkatkan kehadirannya, 
karena calon pelanggan dapat melihat informasi perusahaan secara detail kapan saja (365 hari x 24 jam). 
Biaya pemasaran dapat dikurangi seperti tenaga sales, 
pencetakan brosur dan iklan melalui media tidak perlu dilakukan. 
Oleh karena itu \emph{website} yang dibuat haruslah berisi informasi sesuatu yang menarik.

\section{Maksud dan Tujuan Magang}

Maksud dari kerja magang yang dilaksanakan untuk 
mengembangkan \emph{website} profil perusahaan 
PT. BPS. \emph{Website} profil perusahaan ini bertujuan menjadi media promosi dan meningkatkan \emph{brand}
image dari perusahaan. Memberikan informasi secara detil 
mengenai profil perusahaan dan jasa yang ditawarkan oleh PT. BPS serta 
pengunjung dapat membaca artikel yang berkaitan dengan PT. BPS.

\noindent Tujuan dari kerja magang yaitu:

\begin{enumerate}
\item Menyelesaikan masalah-masalah yang dihadapi di dunia kerja dengan bekal ilmu yang telah dipelajari di kampus.
\item Mengembangkan pengetahuan dan kemampuan mahasiswa melalui pengaplikasian ilmu.
\item Memberikan pelatihan dan pengalaman kerja bagi mahasiswa.
\item \emph{Link and match} pengetahuan yang telah dipelajari di kampus dengan dunia industri.
\end{enumerate}

\section{Waktu dan Ruang Lingkup Pelaksanaan Kerja Magang}

Waktu pelaksanaan kerja magang terhitung dari tanggal 
17 Juni 2019 sampai dengan 18 September 2019 di 
PT. Boga Pangan Sentosa (BPS) yang terletak di 
Perumnas Teluk Jambe, Blok PB No.1 Rt.006/020 Desa Sukaluyu, 
Kec. Teluk Jambe Timur, Karawang-Jawa Barat.

\noindent Prosedur pelaksanaan kerja magang di PT. BPS:
\nolinebreak
\begin{enumerate}
\item Pelaksanaan kerja magang ini dibimbing dan diawasi langsung oleh komisaris yang 
bertanggung jawab sebagai pengembangan \emph{(development)} dari PT. BPS. 
\item Jam kerja pada saat kerja magang, mengikuti jam kantor PT. BPS. 
Senin - Jumat mulai dari jam 09.00-17.00 dengan waktu istirahat antara jam 12.00-13.00. 
\end{enumerate}